% references
% \newcommand{\xref}[1]{\textcolor{textcoloremph}{#1}}

% footnote marks
% \newcommand{\xfootnotemark}[1]{\textcolor{textcoloremph}{\textsuperscript{#1}}}

% list item color changer
% \newcommand{\itemcolor}[1]{
%   \renewcommand{\makelabel}[1]{\color{#1}\hfil ##1}}


% provide file name and path of input files
% https://ctan.org/pkg/currfile
% \usepackage{currfile}
% \ExpandArgs{ne}\newcommand{\configpath}{\currfiledir}


% title mark xtitlemark -> mytitlemark
\newcommand{\mytitlemark}{\textcolor{textcoloremph}{\faLongArrowAltRight}~}



% \newcommand{\cdigits}[1]{\textcolor{niceblue}{#1}}


% \captionsetup{aboveskip=3pt}
% \captionsetup{belowskip=0pt}


\newcommand{\xopt}[1]{\textcolor{uofmbluelighter}{#1}}
\newcommand{\myopt}{\xopt{[Optional]}}

\newcommand{\xtheorem}[1]{{[\textbf{\textcolor{xprimarycolor}{#1}}]}}

\newcommand{\xeqspace}[1]{\quad \text{({#1})}}

\newcommand{\myspace}{\vspace*{\myitemsep}}
\newcommand{\myspacelarge}{\vspace*{\baselineskip}}

\newcommand{\xeq}[1]{\text{\textcolor{textcoloremph}{#1}}}

\newcommand{\myemph}[1]{\textcolor{textcoloremph}{#1}}

\newcommand{\xtableau}[1]{\textcolor{niceblue}{#1}}

\newcommand\twodigits[1]{%
   \ifnum#1<10 0#1\else #1\fi
}


% footnote marks
\newcommand{\xfootnotemark}[1]{\textcolor{textcoloremph}{\textsuperscript{#1}}}


\newcommand{\pd}{\phantom{-}}

\def\hy#1{\parbox{\linewidth}{#1}} % helper for using command several times

\newcommand{\xhy}[1]{\parbox[t]{\linewidth}{\strut#1\strut}\par
}


\newcommand{\matlabfunctionmin}[2]{%
\def \matlabfunctiontext {\detokenize{#1}}%
\def \matlabfunctionurl {#2}%
\texttt{\href{\matlabfunctionurl}{\matlabfunctiontext}}%
}%

\newcommand{\matlabfunction}[2]{%
\def \matlabfunctiontext {\detokenize{#1}}
\def \matlabfunctionurl {#2}
\xmatlabfunction{\href{\matlabfunctionurl}{\matlabfunctiontext}}
}

\newcommand{\xmatlabfunction}[1]{%
\begin{columns}%
\begin{column}{0.08\textwidth}\centering
\vspace*{0.04in}
\if\darkmode1
% \includegraphics[width=\textwidth]{../../../.config-hrg-beamer/figures/matlab-3-dark.pdf}
\includegraphics[width=\textwidth]{{\configpath}figures/matlab-3-dark.pdf}
\else
\includegraphics[width=\textwidth]{{\configpath}figures/matlab-3.pdf}
\fi
\small
\texttt{#1}
\end{column}%
\end{columns}%
}





%----------------------------------------------------------------
% math
%----------------------------------------------------------------
\newcommand{\xub}[1]{\underaccent{\bar}{#1}}
\newcommand{\xob}[1]{\bar{#1}}
\newcommand{\ubar}[1]{\underaccent{\bar}{#1}}
\newcommand{\obar}[1]{\bar{#1}}

\DeclareMathOperator*{\argmax}{arg\,max}
\DeclareMathOperator*{\argmin}{arg\,min}

\DeclarePairedDelimiter\abs{\lvert}{\rvert}
\DeclarePairedDelimiter\norm{\lVert}{\rVert}
\DeclarePairedDelimiter\bracket{[}{]}
\DeclarePairedDelimiter\paren{(}{)}
\DeclarePairedDelimiter\curl{\lbrace}{\rbrace}
\DeclarePairedDelimiter\ceil{\lceil}{\rceil}
\DeclarePairedDelimiter\floor{\lfloor}{\rfloor}


\setcounter{MaxMatrixCols}{20}


\newcommand*{\eqbox}[1]{%
\colorbox{bgcolorlight}{$\displaystyle{#1}$}
}%


\newcommand{\eqrepeat}{\addtocounter{equation}{-1}}

\NiceMatrixOptions{
code-for-first-row = \color{textcoloremph},
code-for-last-row = \color{textcoloremph},
code-for-first-col = \color{textcoloremph},
code-for-last-col = \color{textcoloremph},
}

\makeatother
\let\OldNabla\nabla
\RenewDocumentCommand{\nabla}{e_}{%
    \OldNabla
    \IfValueT{#1}{%
        _{\!#1\,}
    }%
}
\makeatletter

% adds a transparent white rounded rectangle around text
\newcommand{\myoutline}[2]{%
\begin{tikzpicture}
\node [
fill=white,
fill opacity=0.9,
text opacity=1,
rounded corners=0.05in,
#2
] at (0,0) {#1};
\end{tikzpicture}%
}%


%----------------------------------------------------------------
% listings
%----------------------------------------------------------------
%
\lstset{numbers=left,
numberstyle=\tiny,
stepnumber=1,
firstnumber=1,
numbersep=5pt,
language=Matlab,
stringstyle=\ttfamily,
basicstyle=\ttfamily\normalsize,
showstringspaces=false,
commentstyle=\color{nicegreen},
keywordstyle=\color{textcolor},
upquote=true,
breaklines=true,
}



%----------------------------------------------------------------
% references
%----------------------------------------------------------------
% uses shorttitle if available, title otherwise
\DeclareCiteCommand{\citetitle}
  {\boolfalse{citetracker}%
   \boolfalse{pagetracker}%
   \usebibmacro{prenote}}
  {\ifciteindex
     {\indexfield{indextitle}}
     {}%
   \printtext[bibhyperref]{\printfield[citetitle]{labeltitle}}}
  {\multicitedelim}
  {\usebibmacro{postnote}}

% uses title always
\DeclareCiteCommand{\citetitlefull}
  {\boolfalse{citetracker}%
   \boolfalse{pagetracker}%
   \usebibmacro{prenote}}
  {\ifciteindex
     {\indexfield{indextitle}}
     {}%
   \printtext[bibhyperref]{\printfield{title}}}
  {\multicitedelim}
  {\usebibmacro{postnote}}

% formatted url from entry
\DeclareCiteCommand{\citeurl}
  {\boolfalse{citetracker}%
   \boolfalse{pagetracker}%
   \usebibmacro{prenote}}
  {\ifciteindex
     {\indexfield{indextitle}}
     {}%
      \urlfull{\thefield{url}}%
  }
  {}
  {\usebibmacro{postnote}}

% reference slides
\newcommand{\refslides}{%
\section*{\texorpdfstring{\scalebox{0.87485779}{\faBookmark}}{References}}%
\miniframesoff%
\bgroup%
\begin{frame}[c,allowframebreaks]{\mytitlemark References}%
\label{sec:references}%
% \nocite{*}
\printbibliography[heading=none]
\end{frame}%
\egroup%
\miniframeson%
}


%----------------------------------------------------------------
% figures boxes with includegraphics and overpic
%----------------------------------------------------------------
\newtcolorbox{figbox}{%
	boxrule = 1pt,
	boxsep = 1pt,
	left = 3pt,
	right = 3pt,
	breakable,
	colframe = edgecolor,
	enhanced jigsaw,
	parbox = false,
	shadow = {2mm}{-1mm}{0mm}{black!50!white},
	colback = tcolorboxbg,
}

% create length for use the boxes
\newlength{\figurewidth}

\newcommand{\setfigurewidth}[2]{%
\settowidth{\figurewidth}{\includegraphics[scale=#2]{#1}}%
\addtolength{\figurewidth}{20pt}%
}

\newcommand{\myfig}[2]{%
\setfigurewidth{#1}{#2}%
\begin{columns}%
\begin{column}{\figurewidth}%
\begin{figbox}%
\centering%
\includegraphics[scale=#2]{#1}%
\end{figbox}%
\end{column}%
\end{columns}%
}

\newcommand{\myfigcol}[2]{%
\setfigurewidth{#1}{#2}%
\begin{column}{\figurewidth}%
\begin{figbox}%
\centering%
\includegraphics[scale=#2]{#1}%
\end{figbox}%
\end{column}%
}

\NewEnviron{myoverpic}[2]{%
\setfigurewidth{#1}{#2}%
\begin{columns}
\begin{column}{\figurewidth}
\begin{figbox}
\centering
\begin{overpic}[scale=#2]{#1}
\BODY
\end{overpic}
\end{figbox}
\end{column}
\end{columns}%
}

\NewEnviron{myoverpiccol}[2]{%
\setfigurewidth{#1}{#2}%
\begin{column}{\figurewidth}
\begin{figbox}
\centering
\begin{overpic}[scale=#2]{#1}
\BODY
\end{overpic}
\end{figbox}
\end{column}
}

% add the grid to myoverpic
\NewEnviron{myoverpicgrid}[2]{%
\setfigurewidth{#1}{#2}%
\begin{columns}
\begin{column}{\figurewidth}
\begin{figbox}
\centering
\begin{overpic}[scale=#2,grid]{#1}
\BODY
\end{overpic}
\end{figbox}
\end{column}
\end{columns}%
}

% add the grid to myoverpiccol
\NewEnviron{myoverpiccolgrid}[2]{%
\setfigurewidth{#1}{#2}%
\begin{column}{\figurewidth}
\begin{figbox}
\centering
\begin{overpic}[scale=#2,grid]{#1}
\BODY
\end{overpic}
\end{figbox}
\end{column}
}


%----------------------------------------------------------------
% links
%----------------------------------------------------------------
\newcommand{\urlfull}[1]{\textcolor{textcoloremph}{\scalebox{0.9}{\faLink}\,\url{#1}}}

\newcommand{\urlhttps}[1]{\textcolor{textcoloremph}{\scalebox{0.9}{\faLink}\,\href{https://\detokenize{#1}}{\detokenize{#1}}}}

\newcommand{\urlvideo}[1]{\textcolor{textcoloremph}{\scalebox{0.9}{\faVideo}\,\href{https://\detokenize{#1}}{\detokenize{#1}}}}

\newcommand{\myhref}[2]{\textcolor{textcoloremph}{\scalebox{0.9}{\faLink}\,\href{#1}{#2}}}

\newcommand{\myhrefvideo}[2]{\textcolor{textcoloremph}{\scalebox{0.9}{\faVideo}\,\href{#1}{#2}}}

%----------------------------------------------------------------
% parnotes
%----------------------------------------------------------------
\newcommand{\parnotefull}[1]{\parnote{\makebox[\textwidth][l]{#1}}}



%----------------------------------------------------------------
% symbols
%----------------------------------------------------------------
\renewcommand{\qedsymbol}{\textcolor{textcoloremph}{\openbox}}

\newcommand{\myfaEdit}{\scalebox{0.9}{\faEdit}\,}

\newcommand{\myiconstyle}[1]{\textcolor{primarycolor!75}{\scalebox{5}{#1}}}

\newcommand{\mysummaryicon}{\textcolor{primarycolor!75}{\scalebox{5}{\faLightbulb[regular]}}}

\newcommand{\mytaskicon}{\textcolor{primarycolor!75}{\scalebox{5}{\faEdit[regular]}}}

\newcommand{\myreadingicon}{\textcolor{primarycolor!75}{\scalebox{5}{\faBook}}}

\newcommand{\myreferenceicon}{\textcolor{primarycolor!75}{\scalebox{5}{\faBookmark}}}

\newcommand{\myoutlineicon}{\textcolor{primarycolor!75}{\scalebox{5}{\faListOl}}}



%---------------------------------------------------------
% front stuff
%---------------------------------------------------------


\newcommand{\xversion}{\textcolor{textcoloremph}{Version \today{\ }{@}\currenttime}}

\newcommand{\myversion}{%
\begin{textblock}{6}(11.6455,15.5325)
\scriptsize\xversion
\end{textblock}%
}

% https://www.privacypolicies.com/blog/sample-copyright-notice/

\newcommand{\mycopyright}{%
\begin{textblock}{6}(0.6455,15.5325)
\scriptsize\xcopyright
\end{textblock}%
}

%---------------------------------------------------------
% customized bookmarks in the PDF
%---------------------------------------------------------
% https://tex.stackexchange.com/questions/74696
\bookmarksetup{
  open,
  openlevel=3,
  addtohook={%
    \ifnum\bookmarkget{level}=2%
      \bookmarksetup{bold}%
    \fi
  },
}

%---------------------------------------------------------
% \myterm and \mytermslides commands
%---------------------------------------------------------

% https://tex.stackexchange.com/questions/489335
\ExplSyntaxOn
\NewDocumentCommand{\createlist}{m}
 {
  \seq_new:c { g_mo_list_#1_seq }
 }
\NewDocumentCommand{\additem}{mmm}
 {
  \seq_gput_right:cn { g_mo_list_#1_seq } { \__mo_list_do:nn {#2}{#3} }
 }
\NewDocumentCommand{\makelist}{O{}m}
 {
  \group_begin:
  \keys_set:nn { mo/list } { #1 }
  \l__mo_list_pre_tl
  \seq_use:cn { g_mo_list_#2_seq } { }
  \l__mo_list_post_tl
  \group_end:
 }
\keys_define:nn { mo/list }
 {
  pre     .tl_set:N  = \l__mo_list_pre_tl,
  post    .tl_set:N  = \l__mo_list_post_tl,
  command .code:n    = \cs_set:Nn \__mo_list_do:nn { #1 },
  pre     .initial:n = {\begin{itemize}\itemsep=3pt},
  post    .initial:n = {\end{itemize}},
  command .initial:n = \item[\scalebox{0.9}{\faHashtag}] {#1} {#2},
 }
\ExplSyntaxOff

% initialize list
\createlist{mytermlist}

% create counter that will have the frame number
\newcounter{term}
\renewcommand\theterm{\arabic{term}}

% create new empty environment
\newenvironment{term}{}{}

% command to create the labels, targets, and display the term
\makeatletter
\newcommand{\pagetarget}[2]{%
    \phantomsection%
    \setcounter{term}{\insertframenumber-1}%
    \begin{term}%
    \refstepcounter{term}%
    \label{#1-label}%
    \hypertarget{#1}{#2}%
    \end{term}%
}
\makeatother

% myterm formatting
\newcommand{\termformat}[1]{\textit{\textcolor{textcoloremph}{#1}}}

% my term display command
\newcommand{\termdisplay}[1]{\textcolor{textcoloremph}{\scalebox{0.9}{\faHashtag}}\termformat{#1}}

% use md5 hash that better supports equations
% https://tex.stackexchange.com/questions/252566/calculate-the-hash-md5-or-otherwise-of-a-string
\newcommand{\termlabel}[1]{\pdfmdfivesum{\detokenize{#1}}}

% command to define term and add to mytermlist
\newcommand{\myterm}[1]{%
\pagetarget{\termlabel{#1}}{\termdisplay{#1}}%
\additem{mytermlist}{%
\hyperlink{\termlabel{#1}}{\termformat{#1}}}{\text{ is on }\hyperlink{\termlabel{#1}}{Slide \ref{\termlabel{#1}-label}}}%
}

% term slides page
\newcommand{\mytermslides}{%
\section*{\texorpdfstring{\scalebox{1}{\faHashtag}}{Terms}}%
\miniframesoff%
\bgroup%
\begin{frame}[c,allowframebreaks]{\mytitlemark Terms}%
\label{sec:terms}%
\makelist{mytermlist}
\end{frame}%
\egroup%
\miniframeson%
}

%---------------------------------------------------------
% QR code
%---------------------------------------------------------
\newcommand{\myqrcode}[1]{\begin{tcolorbox}[hbox,boxsep=0pt,left=4pt,right=4pt,top=4pt,bottom=4pt]
\qrcode[hyperlink,height=0.75in]{\detokenize{#1}}
\end{tcolorbox}}
